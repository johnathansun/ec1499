%%%%%%%%%%%%%%%%%%%%%%%%%%%%%%%%%%%%%%%%%
% Beamer Presentation
% LaTeX Template
% Version 1.0 (10/11/12)
%
% This template has been downloaded from:
% http://www.LaTeXTemplates.com
%
% License:
% CC BY-NC-SA 3.0 (http://creativecommons.org/licenses/by-nc-sa/3.0/)
%
%%%%%%%%%%%%%%%%%%%%%%%%%%%%%%%%%%%%%%%%%

%----------------------------------------------------------------------------------------
%	PACKAGES AND THEMES
%----------------------------------------------------------------------------------------

\documentclass{beamer}

\usepackage{setspace}

\mode<presentation> {

% The Beamer class comes with a number of default slide themes
% which change the colors and layouts of slides. Below this is a list
% of all the themes, uncomment each in turn to see what they look like.

%\usetheme{default}
%\usetheme{AnnArbor}
%\usetheme{Antibes}
%\usetheme{Bergen}
%\usetheme{Berkeley}
%\usetheme{Berlin}
%\usetheme{Boadilla}
%\usetheme{CambridgeUS}
%\usetheme{Copenhagen}
%\usetheme{Darmstadt}
%\usetheme{Dresden}
%\usetheme{Frankfurt}
%\usetheme{Goettingen}
%\usetheme{Hannover}
%\usetheme{Ilmenau}
%\usetheme{JuanLesPins}
%\usetheme{Luebeck}
\usetheme{Madrid}
%\usetheme{Malmoe}
%\usetheme{Marburg}
%\usetheme{Montpellier}
%\usetheme{PaloAlto}
%\usetheme{Pittsburgh}
%\usetheme{Rochester}
%\usetheme{Singapore}
%\usetheme{Szeged}
%\usetheme{Warsaw}

% As well as themes, the Beamer class has a number of color themes
% for any slide theme. Uncomment each of these in turn to see how it
% changes the colors of your current slide theme.

%\usecolortheme{albatross}
%\usecolortheme{beaver}
%\usecolortheme{beetle}
%\usecolortheme{crane}
%\usecolortheme{dolphin}
%\usecolortheme{dove}
%\usecolortheme{fly}
%\usecolortheme{lily}
%\usecolortheme{orchid}
%\usecolortheme{rose}
%\usecolortheme{seagull}
%\usecolortheme{seahorse}
%\usecolortheme{whale}
%\usecolortheme{wolverine}

%\setbeamertemplate{footline} % To remove the footer line in all slides uncomment this line
%\setbeamertemplate{footline}[page number] % To replace the footer line in all slides with a simple slide count uncomment this line

%\setbeamertemplate{navigation symbols}{} % To remove the navigation symbols from the bottom of all slides uncomment this line
}

\usepackage{enumitem}
\usepackage{empheq}
\usepackage{mdframed}
% \usepackage[margin=1in]{geometry}
\usepackage[dvipsnames]{xcolor}
\usepackage{graphicx} % Allows including images
\usepackage{booktabs} % Allows the use of \toprule, \midrule and \bottomrule in tables

%----------------------------------------------------------------------------------------
%	TITLE PAGE
%----------------------------------------------------------------------------------------

\title[Bernanke-Blanchard, Revisited]{Bernanke-Blanchard, Revisited: \\ An Alternative Model of Pandemic-Era Inflation} % The short title appears at the bottom of every slide, the full title is only on the title page

\author{Jerry Liang and Johnathan Sun} % Your name
\institute[EC 1499] % Your institution as it will appear on the bottom of every slide, may be shorthand to save space
{
Economics 1499\\ % Your institution for the title page
\medskip
\textit{Harvard University} % Your email address
}
\date{December 4th, 2025} % Date, can be changed to a custom date

\begin{document}

\begin{frame}
\titlepage % Print the title page as the first slide
\end{frame}

%----------------------------------------------------------------------------------------
%	PRESENTATION SLIDES
%----------------------------------------------------------------------------------------

%------------------------------------------------


\begin{frame}
\frametitle{Recapping Bernanke-Blanchard}
\begin{enumerate}[label=(\arabic*)]
    \item Bernanke-Blanchard (2023) developed a \textbf{structural VAR model} that decomposed post-pandemic inflation into labor pressures, cost shocks, and expectations.
    \item They attribute most of the COVID inflation to \textbf{cost shocks} (food, energy, and shortages), which they show in their model to be \textbf{transitory}. 
    \item Labor pressures were relatively \textbf{insignificant}, though have recently reasserted itself through the traditional Phillips Curve model.
    \item Their model architecture is summarized in the next slide.
\end{enumerate}
\end{frame}

\begin{frame}
\frametitle{Recapping Bernanke-Blanchard}
\begin{figure}
    \centering
    \includegraphics[width=1\linewidth]{bb_arch.png}
\end{figure}
\end{frame}

\begin{frame}
\frametitle{Issues with Bernanke-Blanchard}
\textbf{What is wrong with their model?}
\begin{enumerate}[label=(\arabic*)]
\item Fails to capture \textbf{supply-side slack dynamics} that affect both wages and prices.
\item Treats shortages as \textbf{exogenous} without decomposing how they arise.
\end{enumerate}
For example, some plausible explanations for why shortages arose:
\begin{enumerate}[label=(\arabic*)]
\item \textbf{Supply-side Restrictions:} Decreased and/or hitting maximum capacity utilization
\item \textbf{Demand-side Shocks:} Increased nominal wages and COVID-era subsidies leading to demand grossly outpacing productive capacity.
\item \textbf{Supply Chain Frictions:} As measured by the NY Fed Global Supply Chain Pressures Index, GSCPI.
\end{enumerate}

\end{frame}

\begin{frame}
\frametitle{Model Architecture: Wages}
Modify:
\begin{align*}
    \log w_t
      &= \underbrace{\mathbb{E}\log p_t}_{\text{Expected price}}
       + \underbrace{\log \omega_t}_{\text{``Aspirational" \textbf{real} wage}} 
       + \beta \underbrace{\log x_t}_{\text{Labor market tightness}} \\
       & \boxed{+ \eta \underbrace{\log u_t}_{\text{Capacity utilization}}}
\end{align*}
Letting aspiration real wage evolve as before:
\begin{align*}
    \log \omega_t
      &= \alpha \underbrace{\log \omega_{t-1}}_{\text{Previous ``aspirational" wage}}
       + (1 - \alpha)\underbrace{\big[\log w_{t-1} - \log p_{t - 1}\big]}_{\text{Previous real wage}}
       + \underbrace{\varepsilon_t}_{\text{Error term}}
\end{align*}
Then:
\begin{align*}
    \log w_t - \log w_{t - 1} &= \left[\mathbb{E}\log p_t - \log p_{t-1}\right] 
    + \alpha \left[\log p_{t-1} - \mathbb{E}\log p_{t-1}\right] \\
    &+ \beta\left[x_t - \alpha x_{t-1}\right] \boxed{+ \eta\left[u_t - \alpha u_{t-1}\right]} + \varepsilon_t
\end{align*}
\end{frame}

\begin{frame}
\frametitle{Model Architecture: Wages}
\textbf{Rationale:} Wages should also depend on supply-side demand---if manufacturing is operating at higher capacity, this should put upwards pressure on wages beyond labor market tightness.
\end{frame}

\begin{frame}
\frametitle{Model Architecture: Prices}
Previously, letting wages $w_t$ be fixed, then prices are set as:
\begin{align*}
    \log p_t = \underbrace{\log w_t}_{\text{Marginal cost of labor}} + \underbrace{z_t}_{{\text{``Shocks" to price-setting.}}}
\end{align*}

So:
\begin{align*}
    \log p_t - \log p_{t-1} = (\log w_t - \log w_{t-1}) + (z_t - z_{t - 1})
\end{align*}

In Bernanke-Blanchard, $z_t$ includes food prices, energy prices, and shortages measured by Google Trends data. We wish to endogenize and decompose these price shock terms.
\end{frame}

\begin{frame}
\frametitle{Model Architecture: Prices}
In our new model,
\begin{align*}
    \log p_t &= \theta_w\underbrace{\log w_t}_{\text{Marginal cost of labor}} + \theta_f \underbrace{\log f_t}_{\text{Marginal cost of food}} + \theta_e \underbrace{\log e_t}_{\text{Marginal cost of energy}} \\
    & \boxed{+\theta_s \underbrace{\log s_t(w_t, u_t, \rho_t, g_t)}_{\text{Supply-side pressure (shortage function)}}}
\end{align*}
where $\rho_t$ is NPGDP and $g_t$ is the GSCPI supply chain pressure index.

In particular, we let this shortage term evolve as follows:
\begin{align*}
    \log s_t &= \phi \underbrace{\log s_{t-1}}_{\text{Residual supply-side pressure}} + \psi \underbrace{[\log w_t - \log \rho_t - \log u_t]}_{\text{Excess demand pressure}} \\
    &+ \xi \underbrace{\log g_t}_{\text{Supply chain pressure}}
\end{align*}
\end{frame}

\begin{frame}
\frametitle{Model Architecture: Prices}
\textbf{Rationale:} We use $\log w- \log \rho- \log u$ as a measure of excess demand since wage $w$ is proportional to consumer demand, and we divide through (subtraction in log space) by the effective capacity of the economy, which is PGDP $\times$ Utilization.
\end{frame}

\begin{frame}
\frametitle{Model Architecture: Inflation Expectations}
Same as Bernanke-Blanchard's linear updates.

Short-run expected inflation:
\begin{align*}
    \mathbb{E}\log p_{t} - \log p_{t - 1} = \delta \mathbb{E}_{t}\pi^{LR} + (1 - \delta)[\log p_{t-1} - \log p_{t - 2}]
\end{align*}
Long-run expected inflation:
\begin{align*}
    \mathbb{E}_{t}\pi^{LR} = \gamma \mathbb{E}_{t-1}\pi^{LR} + (1 - \gamma)[\log p_{t-1} - \log p_{t - 2}]
\end{align*}
\end{frame}

\begin{frame}{Why Our Model?}
\begin{enumerate}[label=(\arabic*)]
\item Captures \textbf{supply-side effects} through \textbf{capacity utilization} (taken as exogenous by Bernanke-Blanchard).
\item Allows for wages to impact inflation beyond just as an input price --- wages can now affect prices through a \textbf{second channel}, its flowthrough to \textbf{excess demand} hitting production constraints.
\item Finally, allows for a more accurate \textbf{decomposition} and understanding of the following questions:
\begin{enumerate}[label=(\roman*)]
    \item What are the \textbf{relative magnitudes} of each factor ($w_t, u_t, g_t$) that led to the shortages and the subsequent inflation?
    \item If a substantive root cause of the post-pandemic shortages were due to demand/wages, how \textbf{transitory/persistent} will this be?
\end{enumerate}
\end{enumerate}
\end{frame}

\begin{frame}
\frametitle{Model Variables}
\begin{table}[h!]
\centering
\footnotesize

\begingroup
\onehalfspacing

\begin{tabular}{ll}
\toprule
\textbf{Variable} & \textbf{Data} \\
\midrule
\textcolor{red}{Prices}        & Change in quarterly CPI (annualized) \\
\textcolor{red}{Wages} & Change in quarterly ECI (annualized) \\
\textcolor{red}{Short-run inflation expectations}         & 1-yr series, Cleveland Fed \\
\textcolor{red}{Long-run inflation expectations}      & 10-yr series, Cleveland Fed \\
\textcolor{red}{Shortages $\dagger$} & Google Trends Data \\
\textcolor{ForestGreen}{Energy prices (relative)} & Ratio of change in CPI energy to change in ECI \\
\textcolor{ForestGreen}{Food prices (relative)} & Ratio of change in CPI food to change in ECI \\
\textcolor{ForestGreen}{Labor market tightness} & $\frac{\text{Vacancies}}{\text{Unemployment}}$ from JOLTS $(v/u)$ \\
\textcolor{ForestGreen}{Capacity utilization $\star$} & FRED Capacity Utilization Total Index \\
\textcolor{ForestGreen}{Nominal potential GDP $\star$} & FRED NGDPPOT \\
\textcolor{ForestGreen}{Supply chain pressure $\star$} & NY Fed GSCPI \\
\bottomrule
\end{tabular}

\endgroup
\caption{Selected variables and their corresponding data series. Red variables are \textcolor{red}{endogenous}, green variables are \textcolor{ForestGreen}{exogenous}. New variables are indicated with $\star$.
\newline
$\dagger$: Note that shortages were \textcolor{ForestGreen}{exogenous} but are now \textcolor{red}{endogenous}.}
\end{table}
\end{frame}

\begin{frame}
\frametitle{Wage Equation Results}
\begin{enumerate}[label=(\arabic*)]
    \item \textbf{LOREM IPSUM}
\end{enumerate}
\end{frame}

\begin{frame}{Wage Equation Predictions}
\textbf{INSERT NEW GRAPH HERE}
\begin{center}
\includegraphics[width=\linewidth]{wage_equation.png}    
\end{center}
\end{frame}

\begin{frame}
\frametitle{Price Equation Results}
\begin{enumerate}[label=(\arabic*)]
    \item \textbf{LOREM IPSUM}
\end{enumerate}
\end{frame}

\begin{frame}{Price Equation Predictions}
\textbf{INSERT NEW GRAPH HERE}
\begin{center}
\includegraphics[width=\linewidth]{price_equation.png}    
\end{center}
\end{frame}

\begin{frame}
\frametitle{Inflation Expectations Results}
\begin{enumerate}[label=(\arabic*)]
    \item \textbf{LOREM IPSUM}
\end{enumerate}

\textbf{INSERT NEW GRAPH HERE}
\begin{center}
\includegraphics[width=0.49\linewidth]{sr.png}
\includegraphics[width=0.49\linewidth]{lr.png}
\end{center}
\end{frame}

\begin{frame}
\frametitle{Impulse Reaction Functions}
\begin{enumerate}[label=(\arabic*)]
    \item \textbf{LOREM IPSUM}
\end{enumerate}

\textbf{INSERT NEW GRAPH HERE}
\begin{center}
\includegraphics[width=0.49\linewidth]{price_impulse.png}
\includegraphics[width=0.49\linewidth]{labor_impulse.png}
\end{center}
\end{frame}

\begin{frame}
\frametitle{COVID Inflation Breakdown}
\begin{enumerate}[label=(\arabic*)]
    \item \textbf{LOREM IPSUM}
\end{enumerate}

\textbf{INSERT NEW GRAPH HERE}
\begin{center}
\includegraphics[width=1\linewidth]{inf_decomp.png}
\end{center}
\end{frame}

\begin{frame}
\frametitle{COVID Wage Breakdown}
\begin{enumerate}[label=(\arabic*)]
    \item \textbf{LOREM IPSUM}
\end{enumerate}

\textbf{INSERT NEW GRAPH HERE}
\begin{center}
\includegraphics[width=1\linewidth]{wage_decomp.png}
\end{center}
\end{frame}

\begin{frame}
\frametitle{Projections}
\begin{enumerate}[label=(\arabic*)]
    \item \textbf{LOREM IPSUM}
\end{enumerate}

\textbf{INSERT NEW GRAPH HERE}
\begin{center}
\includegraphics[width=1\linewidth]{projection.png}
\end{center}
\end{frame}

\begin{frame}
\frametitle{Comparisons with Bernanke-Blanchard}
\begin{enumerate}[label=(\arabic*)]
    \item \textbf{LOREM IPSUM}
\end{enumerate}
\end{frame}

\begin{frame}
\frametitle{Conclusion}
\begin{enumerate}[label=(\arabic*)]
    \item \textbf{LOREM IPSUM}
\end{enumerate}
\end{frame}

\begin{frame}{Appendix: Non-Linearities}
It is also possible to think about this shortage dynamic as an inherently \textbf{non-linear} phenomenon without directly fitting it on Google Trends data. In this case, one could write a model as follows:
\begin{align*}
    \log p_t &= \theta_w\underbrace{\log w_t}_{\text{Marginal cost of labor}} + \theta_f \underbrace{\log f_t}_{\text{Marginal cost of food}} + \theta_e \underbrace{\log e_t}_{\text{Marginal cost of energy}} \\
    & \boxed{+\theta_s \underbrace{\log(\text{SoftReLU}(\log w_t - \log \rho_t - \log u_t)) + \xi \log g_t}_{\text{Supply-side pressure (shortage function)}}}
\end{align*}
The issues with this architecture is as follows:
\begin{enumerate}[label=(\arabic*)]
    \item It is unclear how to determine at what cutoff the non-linearities should take effect without \textbf{overfitting}.
    \item Since this is \textbf{inherently global}, we cannot regress on \textbf{local differences} $\log w_t - \log w_{t-1}$. This means that we must first sum $\sum_{i \leq t} \Delta \log w_i$, plug into SoftReLU, and then take log differences.
    \item Because of the non-linearity, there is \textbf{no clean decomposition} of the impulse reaction functions.
\end{enumerate}
\end{frame}

\end{document}